\documentclass[12pt]{article}
\usepackage[left=2cm,right=2cm,top=2cm,bottom=2cm,bindingoffset=0cm]{geometry}
\usepackage[utf8x]{inputenc}
\usepackage[english,russian]{babel}
\usepackage{cmap}
\usepackage{amssymb}
\usepackage{amsmath}
\usepackage{url}
\usepackage{pifont}
\usepackage{tikz}
\usepackage{verbatim}
\usepackage[most]{tcolorbox}

\usetikzlibrary{shapes,arrows}
\usetikzlibrary{positioning,automata}
\tikzset{every state/.style={minimum size=0.2cm},
initial text={}
}

\definecolor{block-gray}{gray}{0.90} 
\newtcolorbox{myquote}{colback=block-gray,grow to right by=-10mm,grow to left by=-10mm,
boxrule=0pt,boxsep=0pt,breakable} 

\newenvironment{myauto}[1][3]
{
  \begin{center}
    \begin{tikzpicture}[> = stealth,node distance=#1cm, on grid, very thick]
}
{
    \end{tikzpicture}
  \end{center}
}


\begin{document}
\begin{center} {\LARGE Формальные языки, HW05} \end{center}

\bigskip

\begin{enumerate}
\item[2.] \begin{myquote}
$S \to RS|R$\\
$R \to aSb|cRd|ab|cd|\epsilon$
\end{myquote}
Добавим новый начальный нетерминал
 \begin{myquote}
$S_0 \to S$\\
$S \to RS|R$\\
$R \to aSb|cRd|ab|cd|\epsilon$
\end{myquote}
Избавимся от неодиночных терминалов
\begin{myquote}
$S_0 \to S$\\
$S \to RS|R$\\
$R \to ASB|CRD|AB|CD|\epsilon$\\
$A \to a$\\
$B \to b$\\
$C \to c$\\
$D \to d$
\end{myquote}
Удаляем длинные правила
\begin{myquote}
$S_0 \to S$\\
$S \to RS|R$\\
$R \to AQ|CP|AB|CD|\epsilon$\\
$A \to a$\\
$B \to b$\\
$C \to c$\\
$D \to d$\\
$Q \to SB$\\
$P \to RD$
\end{myquote}
Элиминируем $\epsilon$-правила
\begin{myquote}
$S_0 \to S|\epsilon$\\
$S \to RS|R|S$\\
$R \to AQ|CP|AB|CD$\\
$A \to a$\\
$B \to b$\\
$C \to c$\\
$D \to d$\\
$Q \to SB|B$\\
$P \to RD|D$
\end{myquote}
Уничтожаем цепные правила\\
\begin{myquote}
$S_0 \to RS|AQ|CP|AB|CD|\epsilon$\\
$S \to RS|AQ|CP|AB|CD$\\
$R \to AQ|CP|AB|CD$\\
$A \to a$\\
$B \to b$\\
$C \to c$\\
$D \to d$\\
$Q \to SB|b$\\
$P \to RD|d$
\end{myquote} -- НФХ.
\item[3.] КС грамматика, задающая язык
\begin{myquote}
$S \to aaS|aSb|Sbb|ab|aa|bb$
\end{myquote}
Пусть есть слово $a^mb^n$ при $m+n>0, 2|(n+m)$
\begin{itemize}
\item $m = 0 \Rightarrow n>0$ -- чётное. Тогда $\frac{n}{2} - 1$ раз применив правило $S \to Sbb$ и один $S \to bb$ получим\\
$S \to Sbb \to ... \to Sbb...bb \to b^n$ 
\item $n=0$ аналогично
\item $m, n > 0$, чётные. Применим $\frac{n}{2}$ раз правило $S \to Sbb$,  $\frac{m}{2}-1$ раз $S \to aaS$, и одно $S \to aa$
\item  $m, n > 0$, нечётные. Применим $\frac{n-1}{2}$ раз правило $S \to Sbb$,  $\frac{m-1}{2}$ раз $S \to aaS$, и одно $S \to ab$
\end{itemize}
С другой стороны по индукции легко понять, что слово из языка грамматики имеет вид $a^pb^q$, а чётность $p+q$ не меняется при использовании правил, а изначально она чётная, при этом пустого слова нет $\Rightarrow p+q$.
 \end{enumerate}
\end{document}
