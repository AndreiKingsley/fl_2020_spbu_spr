\documentclass[12pt]{article}
\usepackage[left=2cm,right=2cm,top=2cm,bottom=2cm,bindingoffset=0cm]{geometry}
\usepackage[utf8x]{inputenc}
\usepackage[english,russian]{babel}
\usepackage{cmap}
\usepackage{amssymb}
\usepackage{amsmath}
\usepackage{ bbold }
\usepackage{url}
\usepackage{pifont}
\usepackage{tikz}
\usepackage{verbatim}
\usepackage[most]{tcolorbox}

\newcommand{\R}{\mathbb{R}} % set of real numbers
\renewcommand{\inf}{\infty}
\newcommand{\la}{\lambda}

\definecolor{block-gray}{gray}{0.90} 
\newtcolorbox{myquote}{colback=block-gray,grow to right by=-10mm,grow to left by=-10mm,
boxrule=0pt,boxsep=0pt,breakable} 

\newenvironment{myauto}[1][3]
{
  \begin{center}
    \begin{tikzpicture}[> = stealth,node distance=#1cm, on grid, very thick]
}
{
    \end{tikzpicture}
  \end{center}
}


\begin{document}


\bigskip

\begin{enumerate}
\item[2.] Применим несколько раз первой правило, затем несколько раз второе:
\begin{myquote}
$S \to aSbbbb |  T$\\
$T \to aaaTbb | c$
\end{myquote}
Очевидно, грамматика эквивалентна. Докажем однозначность. Пусть мы применили $x$ раз первое правило и $y$ раз второе. Тогда число $a$-шек и $b$-шек в слове будет равно\\
\begin{equation*}
 \begin{cases}
  a = x + 3y\\
  b = 4x + 2y
 \end{cases}
\Rightarrow
 \begin{cases}
   x = \frac{3b-2a}{10}\\
   y = \frac{4a-b}{10}
 \end{cases}
\end{equation*}
Таким образом, однозначно восстановили дерево разбора.
\item[3.] Можно так -- строка, в которой на любом префиксе которой $a$-шек хотя бы в 2 раза больше $b$-шек, а на всей строке ровно в 2.
\item[4.] Второая грамматика порождает язые из строк с $a/c$ на нечётных позициях и $a/b$ на чётных(и при это суммарно нечётной длины). Разделим $F$-ки из первой грамматики на несколько видов(а именно по чётности длины и тому, какая буква(из $b/c$) может стоять на первом а значит всех нечётных местах -- $B_0, B_1 , C_0, C_1$). \\
$a$-шку ставим, если длина нечётная.
 \begin{myquote}
$C_1 \to a$\\
$B_1 \to a$
\end{myquote}
Если добавили в какой-то момент в начало $b$. Это означает, что суффикс после неё($F$ из первой грамматики) -- это строка из $a/c$ на нечётных(причём чётность суффикса инвертируется от текущей). Запишем правила
 \begin{myquote}
$B_0 \to bC_1$\\
$B_1 \to bC_0$
\end{myquote}
Для $c$ правила чуть чуть посложнее, потому что после неё идут две $F$-ки. Но их тоже легко охарактеризовать : первая должна иметь $a/b$ на нечётных, а второй полностью определён первым -- если она окончилась на $a/b$, то есть первый нечётный длины, то второй начинается $a/c$, иначе оба с $a/b$ на первом. Суммарная длина должна иметь чётность инвёрнутую к входной. Таким образом следующие правила
 \begin{myquote}
$C_0 \to cB_0B_1|cB_1C_0$\\
$C_1 \to cB_0B_0 |cB_1C_1$
\end{myquote}


Итого грамматика:

\begin{myquote}
$S \to C_1$\\
$B_0 \to bC_1$\\
$B_1 \to bC_0 | a$\\
$C_0 \to cB_0B_1|cB_1C_0$\\
$C_1 \to cB_0B_0 |cB_1C_1 | a$
\end{myquote}
\end{enumerate}

\end{document}
